%!Tex Root=**/conspect.tex
%!TeX program = xelatex
\Subsection{Напоминание}
Говорим про комплексные числа: $\CC \approx \R^2$: $(x, y) \in \R^2 \leadsto z = x + iy. i \in \CC, i^2 = -1, i \leadsto (0, 1), z = x+iy, w = a+ib, z \cdot w = xa - yb + i(xb+ya)$.
\begin{properties}
    \begin{enumerate}
        \item $|z| = \sqrt{x^2+y^2}, \overline{z} \coloneqq x - iy, z \cdot \overline{z} = |z|^2$.
        \item $\Re z = x$ --- вещественная часть,
        \item  $\Im z = y$ --- мнимая часть,
        \item $\Re z = \frac{1}{2} \left( z + \overline{z} \right), \Im z = \frac{1}{2i}\left( z - \overline{z} \right)$.
    \end{enumerate}
\end{properties}
\begin{definition}
    Полярная запись комлексного числа $z = r \left( \cos \vphi + i \sin \vphi \right) = e \cdot e^{i\vphi}$. Умножение: $r_1r_2\left( \cos (\vphi + \psi) + i\sin(\vphi + \psi) \right)$
\end{definition}
\Subsection{Аналитические функции}
\begin{definition}
    Степенной ряд --- это ряд вида $\sum\limits_{n \ge  0} a_nz^n, a_n \in \CC$.
    
    Радиус сходимости $\sum\limits_{n \ge 0} a_nz^n$ --- это $R \in \left[ 0, +\infty \right]\!: R^{-1} = \limsup\limits_{n \to \infty} |a_n|^{\frac{1}{n}}$.

    $R = \sup \left\{ r\!: |a_n r^n| \right\}$.
\end{definition}
\begin{statement}
    \slashn
    \begin{enumerate}
        \item $\sum\limits_{n \ge 0} a_n z^n$ --- сходится абсолютно при $|z| < R$ и расходится при  $|z| > R$.
        \item  $r < R$,  $\sup\limits_{|z|\le  r} \left| \sum\limits_{n \ge  0} z_n z^n \right| < \infty$.
    \end{enumerate}
\end{statement}
\begin{proof}
    Смотри конспект прошлого года.
\end{proof}
\begin{statement}
    \slashn
     \begin{enumerate}
         \item Пусть $R$ --- радиус сходимости  $\sum\limits_{n \ge  0} a_nz^n$, тогда $R$ --- радиус сходимости и для ряда  $\sum\limits_{n \ge 1} n a_n z^{n-1}$.
         \item $\sum\limits_{n \ge 0} a_n z^n \cdot \sum\limits_{m \ge  0} b_n z^n = \sum\limits_{k \ge 0} \sum\limits_{n + m = k} a_nb_m \cdot z^k$. Верно $\forall z\!: |z| < R$.
    \end{enumerate}
\end{statement}
\begin{example}
    $\exp(z) = \sum\limits_{n \ge  0} \frac{z^n}{n!}, R = +\infty.$
\end{example}
\begin{example}
    $\frac{1}{1-z} = \sum\limits_{n \ge  0} z^n, R = 1$.
\end{example}
\begin{definition}
    $\Omega \subset \CC$ --- открытое,  $f\!: \Omega \to \CC$ аналитична, если  $\forall z_0 \in \Omega \exists r > 0\!: \forall z, |z-z_0| < r \implies f(z) = \sum\limits_{n \ge 0} a_n(z-z_0)^n$.
\end{definition}
\begin{statement}
    $f, g$ --- аналитические функции на  $\Omega$, то  $f + g$ --- аналитическая.
\end{statement}
\begin{proof}
    Очевидно.
\end{proof}
\begin{example}
    \begin{enumerate}
        \item $f \in \CC[z] \implies f$ --- аналитическая на  $\Omega = \CC$.
        \item Рациональные функции аналитичны там, где они определены.
    \end{enumerate}
\end{example}
\begin{remark}
    $\mathcal{A}(\Omega) = \left\{ f\!: \Omega \to \CC \mid f\text{ --- аналитическая} \right\} $, тогда $\mathcal{A}$ --- кольцо.
\end{remark}
\Subsection{Голоморфные функции}
\begin{definition}
    $\Omega \in \CC$ --- область, если  $\Omega$ --- открытое, непустое, связное.
\end{definition}
\begin{definition}
    $f\!: \Omega \in \CC$, тогда  $f$ имеет  в  $z_0 \in \Omega \iff \exists \lim\limits_{h \to 0} \frac{f(z_0 + h) - f(z_0)}{h} \eqqcolon f'(z_0) \iff \exists \alpha \in \CC\!: f(z_0 + h) = f(z_0) + \alpha \cdot h + o(|h|), h \to 0$.
\end{definition}
\begin{definition}
    $f\!: \Omega \to \CC$ --- голоморфная, если  $\exists f'(z)\ \forall z \in \Omega$.
\end{definition}
\begin{properties}
    \begin{enumerate}
        \item $f, g$ --- голомофрные функции на  $\Omega$, то  $f+g, f \cdot g$ --- голомофрные, $\frac{f}{g}$ --- голомофорна там, где $g \neq 0$.
    \end{enumerate}
\end{properties}
\begin{proof}
    $f(z+h) \cdot g(z+h) = \left( f(z) + f'(z)h + o(|h|) \right) \left( g(z) + g'(z) h + o(|h|) \right)$ аааааааааааааааааааааааааааааа
\end{proof}
\begin{example}
    \begin{enumerate}
        \item $f \in \CC[z] \implies f$ --- голоморфна. Достаточно проверить для  $f = 1, f = z$.  $f = 1 \implies f' = 0$,  $f = z \implies f' = 1$.
        \item  $f(z) = \overline{z}$, тогда $f$ --- не голоморфна. Посмотрим в нуле: $f'(0) = \lim\limits_{h \to 0} \frac{f(h) - f(0)}{h} = \frac{\overline{h}}{h}$. $h = \eps \in \R$. Тогда предел  $1$, при  $h = i\eps$ получаем предел  $-1$.
        \item $f(z) = \sum\limits_{n \ge 0} a_n z^n$, $R$ --- радиус сходимости,  $R > 0$, тогда  $f$ голоморфна в  $\Omega = D_R = \left\{ z\!: |z| < R \right\} $, причем $f'(z) = \sum\limits_{n \ge  1}na_nz^{n-1}$.
            \begin{proof}
               1. TODO.

               $\left| \frac{(z+h)^n - z^n}{n} - n z^{n-1} \right| \le  n(n-1)(|z| + |h|)^{n-2} \cdot |h|, n \ge 2$.

               $\left| \frac{(z+h)^n - z^n}{h} - n \cdot z^{n-1} \right| = \left| (z+h)^{n-1} + \ldots + z^{n-1} - n z^{n - 1} \right| = \left| (z+h)^{n-1} - z^{n-1} + (z+h)^{n-2}z - z^{n-1} + \ldots + z^{n-1} - z^{n-1} \right| \le  \sum\limits_{k = 0}^{n-1} |z|^k \cdot |(z+h)^{n-1-k}-z^{n-1-k}| \le  n(n-1)(|z|+|h|)^{n-2}|h|$.

               Покажем, что $f'(z) = \sum\limits_{n \ge  1}na_n z^{n-1}$:
               \begin{align*}
                   \left| \frac{f(z+h) - f(z)}{h} - \sum\limits_{n \ge  1}n a_n \cdot a^{n-1} \right| &= \left| \sum\limits_{n\ge 2} a_n\left( \frac{(z+h)^n - z^n}{h} - nz^{n-1} \right)  \right| \le \\ &\le \left( \sum\limits_{n \ge  2} |a_n| n(n-1)(|z| + |h|)^{n-2}\right) |h| \xrightarrow{h \to 0} 0
               .\end{align*}
            \end{proof}
    \end{enumerate}
\end{example}
\begin{consequence}
    Аналитические функции --- голоморфны. Обратное утверждение тоже верно.
\end{consequence}
\Subsection{Уравнение Коши-Римана}
Рассмотрим $f(z) = u(z) + i \cdot v(z), u, v \in \R, u = \Re f, v = \Im f$. Можно посмотреть на  $f$ как на  $f\!: \Omega \to \R^2$,  $(x, y) \mapsto \left(u(x, y), v(x, y)\right)$
 \begin{statement}
     $f\!: \Omega \to \CC$ --- голоморфная, тогда  $f$ дифференцируема как функция из  $\R^2$ в  $\R^2$, и матрица Якоби  $f$ имеет вид  $\begin{pmatrix} u_x & u_y \\ v_x & v_y \end{pmatrix} $
\end{statement}
\begin{proof}
    $z = x+iy, h = h_1+ih_2$.
    $f(z+h) = f(x+h, i(y+h_2)) = f(z) + f'(z) \cdot (h_1 + ih_2) + o(|h|)$, $h+ih_2 \mapsto f'(z) \cdot (h_1 + ih_2)$ --- линейное отображение $\R^2 \to \R^2$.
\end{proof}

$f\!: \Omega \in \CC$ --- голоморфная,  $z \in \Omega, \eps \in \R$.  $f'(z) = \lim\limits_{\eps \to 0} \frac{f(z+\eps) - f(z)}{\eps} = \lim\limits_{\eps \to 0} \frac{f(z+i\eps) -f(z)}{iz}$. Тогда, если $z = x+iy, \lim\limits_{\eps \to 0} \frac{f(x+\eps + iy) - f(x+iy)}{\eps} = \frac{\partial f}{\partial x}(z)$. По $y$ получается предел  $-i \frac{\partial f}{\partial y}(z)$. 

То есть $\frac{\partial f}{\partial x}(z) = -i \frac{\partial f}{\partial y}(z) \iff \begin{cases}
    \frac{\partial u}{\partial x} = \frac{\partial v}{\partial y} \\ 
    \dots
\end{cases}$

\begin{definition}
$\frac{\partial f}{\partial x} = \frac{1}{\eps}\left( \frac{\partial f}{\partial x} + i \frac{\partial f}{\partial y} \right)$. Что-то. 
\end{definition}
\begin{lemma}
    $f\!: \Omega \to \CC, \Omega$ --- область.
     $f$ голоморфна  $\iff f$ дифференцируема и $\frac{\partial f}{\partial \overline{z}} = 0$.
\end{lemma}
\begin{proof}
    \begin{itemize}
        \item $\implies$ проверили выше.
        \item $\Leftarrow$ $z = x + iy, h = h_1 + ih_2, f = u + iv$.
            \begin{align*}
                f(z+h) = f(x + h_1 + i(y+h_2)) = f(x+iy) + \frac{\partial f}{\partial x}(z) h_1 + \frac{\partial f}{\partial y}(z) \cdot  h_2 + o(|h|) = f(z) + (u_x + iv_x)h_1 + (u_u + iu_y)h_2 + o(|h|) = f(z) + (u_x + iv_x)h_1 + (-v_x + i u_x)h_2 + o(|h|) = f(z) + (u_x + iv_x) (h_1 + ih_2) + o(|h|) = f(z) + (u_x i v_x) h + o(|h|)
            .\end{align*}
    \end{itemize}
\end{proof}
\Subsection{Первообразная голоморфной функции}
\begin{definition}
   $\Omega$ --- область,  $f\!: \Omega \to \CC$, тогда $F\!: \Omega \to \CC$ --- первообразная  $f$, если  $F'(z) = f(z)\ \forall z \in \Omega$.

   В частности, $F$ --- голоморфна.
\end{definition}
\subsubsection{Интеграл вдоль пути}
\begin{definition}
    Путь --- непрерывное отображение $z\!: \left[ a, b \right] \to \CC$.
\end{definition}
\begin{definition}
    Путь гладкий, если $\forall t \in \left( a, b \right) \exists z'(t)$ непрерывно ограничена

    Кусочно гладкий, если $\exists t_1, \ldots, t_n \in \left( a,b \right)\!: z'(t)$, если $t \neq t_i$.
\end{definition}

\begin{definition}
    Пути $z_1: [a, b] \to \CC, z_2: [c,d] \to C$ эквиваленты, если они отличаются заменой параметризации, то есть $\exists \vphi\!:[a, b] \to [c,d]$ биекция.
\end{definition}
\begin{definition}
    Контур --- класс эквивалентности путей.
\end{definition}
\begin{definition}
    Пусть $\gamma$ --- контур, заданный путем  $z\!: [a, b] \to \CC$, тогда контур  $C$ обратной ориентации --- это контур, заданный путем  $\widetilde{z}\!: [-b, -a] \to CC, \widetilde{z}(t) = z(-t)$.
\end{definition}
\begin{definition}
    Длина пути, это $\int\limits_a^b \sqrt{x'(t)^2 + y'(t)^2 \dd{t}}$
\end{definition}
\begin{definition}
    $f\!: \Omega \to \CC$ --- непрерывна, тогда интеграл вдоль пути  $\gamma$, заданный $z\!: [a, b] \to \CC$, это  $\int\limits_{\gamma} f(z)\dd{z} = \int\limits_a^b f(z(t))z'(t)\dd{t}$.
\end{definition}
\begin{statement}
    $\int\limits_{\gamma} f(z) \dd{z}$ не зависит от выбора пути  $\gamma$.
\end{statement}
\begin{proof}
    $\vphi\!: [c,d] \to [a, b], z_1\!: [c,d] \to \CC, z_1(t) = z(\vphi(t)), z_1'(t) = z'(\vphi(t))\vphi'(t)$.

    $\int\limits_c^d f(z_1(t))z_1'(t) \dd{t} = \int\limits_c^d f(z(\vphi(t)) z'(\vphi(t)) \vphi'(t) \dd{t} \overset{s=\vphi(t)}{=} \int\limits_a^b f(z(s))z'(s) \dd{s}$.
\end{proof}
\begin{consequence}
    $\gamma$ --- контур, то  $\int_{\gamma} (z) \dd{z}$ можно определить как интеграл по пути, параметризующим этот контур.
\end{consequence}
\begin{example}
    $f(z) = z^n$. Путь  $z\!: [0, 2\pi] \to \CC, z(t) = e^{it}$. Соответствует контуру $\gamma$ --- окружность.

    $\int\limits_{\gamma}z^n \dd{z} = \int\limits_0^{2\pi} e^{i nt} \cdot i e^{it} \dd{t} = i\int\limits_0^{2\pi}e^{i(nt)^t}\dd{t} = i\int\limits_0^{2\pi}\left( \cos((n+1)t) + i\sin\left( (n+1)t \right)  \right) \dd{t} = \begin{cases}
        2\pi i, n = -1\\
        0, n \neq -1
    \end{cases}$
\end{example}
\begin{statement}
    $\gamma$ --- контур,  $\widetilde{\gamma}$ --- контур с обратной ориентацией, тогда  $\int\limits_{\gamma}f(z) \dd{z} = -\int\limits_{\widetilde{\gamma}} f(z) \dd{z}$
\end{statement}
\begin{proof}
    $z\!: [a, b] \to \CC$ --- это параметризация  $\gamma$,  $\widetilde{z}: [-b, -a] \to \CC$ параметризация  $\widetilde{\gamma}$.
\[
    \int\limits_a^b f(z(t))z'(t)\dd{t} \overset{s = -t}{=} f(z(-s)) z'(-s)(-\dd{s}) = -\int\limits_a^b f(\widetilde{z}(s))\widetilde{z}(s)\dd{s} = \ldots
.\] 
\end{proof}
\begin{statement}
    $\gamma$ --- контур, тогда  $\left| \int\limits_\gamma f(z) \dd{z} \right| \le length(\gamma) \max\limits_{z \in \gamma} |f(z)|$.
\end{statement}
\begin{proof}
    Расписать интеграл, ограничить $f(z)$ максимумом.
\end{proof}
\begin{statement}
    $f\!: \Omega \to \CC$, пусть  $\exists F\!: \Omega \to \CC$ --- первообразная $f$. Тогда если  $z\!: [a, b] \to \Omega$ --- путь, задающий контур $\gamma$, то
     \[
         \int_\gamma f(z)\dd{z} = F(z(b)) - F(z(a))
    .\] 
\end{statement}
\begin{proof}
    $\frac{\dd{}}{\dd{t}} F(z(t)) = F'(z(t)) \cdot z'(t)$, и правда: $\eps > 0, F(z(t + \eps)) = F(z(t) + z'(t) \eps + o(\eps)) = F(z(t)) + F'(z(t)) (z(t) \eps + o(\eps)) + o(\eps) = F(z(t)) F'(z(t)) \cdot z'(t) \eps + o(\eps)$.

    $\int\limits_{\gamma} f(z) \dd{z} = \int\limits_a^b f(z(t))z'(t)\dd{t} = \int\limits_a^b F'(z(t)) \cdot z'(t) \dd{t} = \int\limits_a^b \frac{\dd{}}{\dd{t}} F(z(t)) \dd{t} = \int\limits_a^b \frac{\dd{}}{\dd{t}} \Re F(z(t))\dd{t} + i\int\limits_a^b \frac{\dd{}}{\dd{t}} \Im F(z(t)) \dd{t} = F(z(b)) - F(z(a))$.
\end{proof}
\begin{theorem}
    $\Omega$ --- область,  $f\!: \Omega \to \CC$ --- голоморфная функция.  $T \subset \Omega$ --- контур, совпадающий с границей треугольника, лежащего в  $\Gamma$. Тогда  $\int_T f(z)\dd{z} = 0$.
\end{theorem}
\begin{proof}
    Картинка! $\int\limits_T f(z) \dd{z} = \int\limits_{T_1^{(1)}} f(z) \dd{z} + \int\limits_{T_2^{(1)}} f(z) \dd{z} = \int\limits_{T_3^{(1)}}$.

    Картинка про аддитивность.

    Тогда по индукции определим $T_i^{(k)}$, для каждого $k \ge  1$ $\left| \int\limits_T f(z)\dd{z} \right| = \left| \sum\limits_{j=1}^{4^n} \pm \int\limits_{T_j^{(k)}} f(z) \dd{z} \right| \le  4^k \cdot \max_j \left|\int\limits_{T_j^{(k)}} f(z) \dd{z}\right|$

    $f(z) = f(z_0) + f'(z_0) \cdot (z - z_0) + o(z- z_0)$. Тогда $\int\limits_{T_j^{(k)}} f(z)\dd{z} = \underbracket{\int\limits_{T_j^{(k)}} f(z_0) \dd{z}}_{ = 0} + \underbracket{\int\limits_{T_j^{(k)}} f'(z_0) (z-z_0) \dd{z}}_{= 0} + \int\limits_{T_j^{(k)}} o(z-z_0)\dd{z} \implies \left| \int\limits_{T_j^{(k)}} f(z) \dd{z} \right| \le  \max_{z \in T_j(k)} \left| f(z) - f(z_0) - f'(z_0)(z-z_0) \right| \cdot Perim(T_j^{(k)}) \le o(2^{-k}diam(T)) \cdot 2^{-k} Perimtetr(T) = o(4^{-k})$.

    А значит, интеграл по контуру равен 0.
\end{proof}
\begin{definition}
    $\Omega$ называется односвязной, если  $\forall \gamma$ --- замкнутый (такого, что  $\gamma \subset \Omega$), ограниченная компонента связности  $\CC \setminus \gamma$ тоже содержится в $\Gamma$.
\end{definition}
\begin{theorem}
    $\Omega$ --- односвязная область,  $f\!: \Omega \to \CC$ --- голоморфная функция, тогда  $\exists F\!: \Omega \to \CC$ --- первообразная  $f$,  $F'(z) = f(z)\ \forall z \in \Omega$.
\end{theorem}
\begin{proof}
    $w_0 \in \Omega$ --- фиксирована.  $\forall w$ построим путь  $\gamma_w$ из  $w_0$ в  $w$, который движется либо вертикально, либо горизонтально, а также не самопересекается. 

    $F(w) \coloneqq \int\limits_{\gamma_w}f(z) \dd{z}$. А дальше в следующей серии!
\end{proof}
\begin{consequence}
    Если $\gamma$ --- замкнутый контур,  $f$ --- голоморфная функция в односвязной области  $\Omega$,  $\gamma \subset \Omega \implies \int\limits_\gamma f(z)\dd{z} = 0$.
\end{consequence}
\begin{definition}
    Петля --- непрерывный образ окружности, то есть отображение вида $z\!: [a, b] \to \CC\ z(a) = z(b)$.
\end{definition}
\begin{definition}
    Петля простая, если она не самопересекается, то есть $\forall x \in [a, b], y \in (x, y)\!: z(x) \neq z(y)$.
\end{definition}
\begin{theorem}[Теорема Жордана]
    Плоскость разбивается простой петлей ($\gamma \in \CC$) на ограниченное связное множество $\Omega_1$ и неограниченное связное множество $\Omega_2$.
\end{theorem}
\begin{definition}
    $\Omega_1$ --- это жорданова область, $\partial \Omega_1 \coloneqq \gamma$, ориентированная против часовой стрелки.
\end{definition}
\begin{theorem}[Гурса]
    $T$ --- треугольник,  $f\!: \Omega \to \CC$ --- голоморфная,  $T \subset \Omega$. Тогда  $\int\limits_{\partial T} f(z) \dd{z} = 0$.
\end{theorem}
\begin{definition}
    $\gamma$ --- координатный путь (петля), если  $\gamma$ составлена из конечно числа вертикальных и горизонтальных отрезков.
\end{definition}
\begin{lemma}
    Пусть $\Omega$ --- односвязная область,  $\gamma \subset \Omega$ --- координатная петля, $f\!: \Omega \to \CC$ --- голоморфная. Тогда:
    \[
        \int\limits_\gamma f(z)\dd{z} = 0
    .\] 
\end{lemma}
\begin{proof}
    Упражнение. Можно разбить наш контур на прямоугольники. Каждый прямоугольник --- на треугольники (построить триангуляцию). Дальше теорема Гурса.
\end{proof}
\begin{theorem}
    $\Omega$ --- односвязная область,  $f\!: \Omega \to \CC$ --- голоморфная, то  $\exists$ первообразная  $f$, то есть  $F\!: \Omega \to \CC\!: F'f = f$.
\end{theorem}
\begin{proof}
    Возьмем $w_0 \in \Omega$.  $\forall w \exists $ координатный путь  $\gamma_w$, соединяющий  $w_0$ и  $w$,  $\gamma_w \subset \Omega$. Тогда возьмем  $F(w) \coloneqq \int_{\gamma_w} f(z) \dd{z}$.
    \begin{enumerate}
        \item $F(w)$ не зависит от выбора  $\gamma_w$. Если  $\gamma_w^1, \gamma_w^2$ --- два координатных пути, соединяющих  $w_0$ и  $w$, то  $\gamma = \gamma_w^1 \cup \gamma_w^2$ --- координатная петля  $\implies$ разность интегралов равна нулю.
        \item Проверим, что  $F'(w) = f(w)$.  $F'(w) = \lim_{h \to 0} \frac{F(w+h) - F(w)}{h}$.  $F(w + h) = \int\limits_{\gamma_{w+h}}f(z) \dd{z} = \int\limits_{\gamma_w} f(z) \dd{z} + \int\limits_{\gamma_h}f(z) \dd{z} = F(w) + \int\limits_{\gamma_h} f(z) \dd{z}$.

            $h^{-1}\left( F(w+h) - F(w) \right) = h^{-1}\int\limits_{\gamma_h}f(z)\dd{z} = h^{-1}\int\limits_0^1 f(w + th)h\dd{t} = \int_0^1 f(w + th) \dd{t} \xrightarrow{h \to 0} f(w)$.
    \end{enumerate}
\end{proof}
\subsubsection{Формула Коши}
\begin{theorem}
    $\Omega \subset \CC$ --- область,  $f\!: \Omega \to \CC$ --- голоморфная функция. Пусть  $w_0 \in \Omega, r > 0\!: \overline{B}(w_o, r) \subset \Omega$. Тогда:
    \[
        \forall z \in B(w_0, r) = f(z) = \frac{1}{2\pi} \int\limits_{\mathclap{|w-w_0|=r}} \frac{f(w)}{w - z} \dd{w}
    .\] 
    Окружность против часовой стрелки ориентирована!
\end{theorem}
\begin{proof}
    Картинка! $\int\limits_\gamma \frac{f(w)}{w-z}\dd{w} = 0$, так как $\gamma$ замкнутый, а $\frac{f(w)}{w-z}$ --- это голоморфная функция по $w$. 

    $\int\limits_{\gamma} \frac{f(w)}{w-z}\dd{w} = \int\limits_{l} \frac{f(w)}{w-z}\dd{w} - \int\limits_{|w-z|=\eps} \frac{f(w)}{w-z}\dd{w} - \int\limits_{l}\frac{f(w)}{w-z}\dd{w} + \int\limits_{|w-w_0|=r}\frac{f(w)}{w-z}\dd{w}$. Что понятно равно $\int\limits_{|w-w_0|=r} \frac{f(w) \dd{w}}{w-z} = \int\limits_{|w-z|=\eps} \frac{f(w)}{w-z}\dd{w} = \int\limits_{|w-z|=\eps} \frac{f(z) + f(w) - f(z)}{w-z}\dd{w} = f(z) \int\limits_{|w-z|=\eps} \frac{\dd{w}}{w-z} + \int\limits_{|w-z|=\eps} \frac{f(w) - f(z)}{|w-z|=\eps}\dd{w}$. Первое слагаемое $f(z) \cdot 2\pi i$, а второе можно оценить $| \circ | \le  \max_{|w-z|=\eps} \left| \frac{f(w) - f(z)}{w - z} \right| 2\pi \eps \le (|f'(z)| +1) 2\pi \eps$
\end{proof}
\begin{consequence}
    \begin{enumerate}
        \item Голоморфные функции --- аналитичны! Пусть $f!\: \Omega \to \CC$ --- голоморфная,  $\overline{B}(w_0, r) \subset \Omega, z \in B(w_0, r)$. Тогда $f(z) = \frac{1}{2\pi i} \int\limits_{|w-w_0|=r} \frac{f(w)}{w-z} \dd{w} = \frac{1}{2\pi i} \int\limits_{|w-w_0|=r} \frac{f(w)\dd{w}}{w - w_0 - (z-w_0)} = \frac{1}{2\pi i} \int\limits_{|w-w_0| = r}f(w) \sum\limits_{n \ge 0} \frac{(z-w_0)^n}{(w-w_0)^{n+1}} \dd{w} = \sum\limits_{n \ge 0} (z-w_0)^n \frac{1}{2 \pi i} \int_{|w-w_0|=r} \frac{f(w) \dd{w}}{(w-w_0)^{n+1}}$.
    \end{enumerate}
    То есть $\forall z \in B(w_0, r)\ f(z) = \sum\limits_{n \ge  0} a_n\left(z - w_0  \right)^n$.
    \item Теорема Луивилля: $f\!: \CC \to \CC$ --- голоморфная и ограниченная. Тогда $f \equiv const$.
         \begin{proof}
             Заметим, что $f'(z) = \frac{1}{2\pi i} \int\limits_{|w-z|=R} \frac{f(w)}{(w-z)^2} \dd{z}$. Тогда если $|f(w)| \le C \forall w$. Тогда $|f'(z)| \le  \frac{1}{2\pi} \cdot \frac{C}{R^2} 2 \pi R \ldots$.
         \end{proof}
     \item Основная теорема алгебры: $P \in \CC[z], \deg P = n$, тогда  $P$ имеет  $n$ корней в  $\CC$.
          \begin{proof}
              Докажем, что при $n \ge  1$ есть хотя бы один корень. Пусть $P(z) = \sum\limits_{i=0}^n a_iz^i$. Тогда если взять  $|f(z)| = \frac{1}{z^n (a_n + a_{n-1}\frac{1}{z} + \ldots + a_0 \frac{1}{z^n})}$
         \end{proof}
     \item Теорема единственности. $f\!: \Omega \to \CC$ --- голоморфная,  $\Omega$ --- область,  $f \centernot \equiv 0$. Тогда  $\left\{ z \in \Omega  \mid f(z) = 0 \right\}$ дискретно (то есть не имеет точек сгущения в $\Omega$.
          \begin{proof}
             Пусть $z_0, z_1, z_2, \ldots \in \Omega$, такое что $f(z_k) = 0\ \forall k \ge  0$, $z_k \to z_0, z_k \neq z_0, \forall k \ge  1$.

             Пусть  $r > 0\!: \overline{B}\left( z_0, r \right) \subset \Omega, f(z) = \sum\limits_{n \ge 0} a_n (z-z_0)^n, \exists d\!: a_d \neq 0$.

             $f(z) = (z - z_0)^d \sum\limits_{n \ge  d} a_n(z-z_0)^{n-d}= \left( z-z_0 \right)^d g(z)$. $g$ --- голоморфная в  $B(z_0, r), g(z_0) \neq 0 \implies \exists N \forall n \ge  N\ g(z_n) \neq 0 \implies f(z_n) \neq 0?!$
         \end{proof}
\end{consequence}

\begin{definition}
    $\Omega$ --- односвязное область,  $\partial \Omega$ --- кусочно гладкий путь.  $f\!: \partial \Omega \to \CC$ непрерывна, голоморфна в  $\Omega$.

    Тогда  $\int\limits_{\partial \Omega} f(z) \dd{z} = 0$.
\end{definition}

Пояснение: $r_n\!: [0, 1] \to \Omega$ --- кусочно гладкий замкнутый путь  $\gamma_n$.  $r_n \to r \implies 0 = \int\limits_{\gamma_n} f(z) \dd{z} \to \int\limits_{\partial \Omega}f(z) \dd{z} = 0$

\begin{definition}
    $\Omega$ --- односвязная область,  $\partial \Omega$ --- кусочно гладкий путь,  $z_1, \ldots, z_n \in \Omega$, $\eps > 0\!: \overline{B}(z_k, \eps) \subset \Omega\ \forall k=1..n$, $C_\eps(z_k) = \left\{ z\!: |z-z_k| = \eps \right\} \implies \exists$ кусочно гладкий путь $r_k\!:[0, 1] \to \CC\!: r_k(0) \in C_\eps(z_k), r_k((0, 1)) \subset \Omega \setminus \bigcup \overline{B}(z_k, \eps), r_k([0, 1]) \cap r_j([0, 1]) = \emptyset\ k \neq j$
\end{definition}

\begin{definition}
    $\Omega$ --- область, $z_0 \in \Omega$,  $f\!: \Omega \setminus \left\{ z_0 \right\} \to \CC$ голоморфная. Тогда $z_0$ --- особенность  $f$. Различают 3 типа особенностей:
    \begin{itemize}
        \item Устранимая $\iff$ $f$ ограничена в  $B(z_0, \eps) \setminus \left\{ z_0 \right\} $ для некоторого $\eps > 0$.
        \item Полюс  $\iff$ $h(z) = \frac{1}{f(z)}$ определена и голомофрна в $B(z_0, \eps)$ для некоторого $\eps > 0$.
        \item Существенная  $\iff$ не 1 или 2.
    \end{itemize}
\end{definition}
\begin{theorem}[Об устранимой особенности]
    Пусть $\Omega$ --- область,  $z_0 \in \Omega$,  $f\!: \Omega \setminus \left\{ z_0 \right\} \to \CC$ --- голоморфна и $z_0$ --- устранимая особенность  $f$. Тогда  $\exists \lim\limits_{z \to z_0} f(z) = f(z_0)$ и $f$ является голоморфной в  $\Omega$.
\end{theorem}
\begin{proof}
    Возьмем  $\eps > 0\!: \overline{B}(z_0, \eps) \subset \Omega$. Рассмотрим $F(z) = \frac{1}{2 \pi i} \int\limits_{|\xi - z_0| = \eps} \frac{f(\xi) \dd{\xi}}{\xi - z}$. 

    Докажем, что $F(z) = f(z)\ \forall z \in B(z_0, \eps) \setminus \{z_0\}$ 

    $\gamma$ --- контур  $\partial \left( B(z_0, \eps) \setminus \left( \overline{B}(z_0, \delta) \cup \overline{B}(z, \delta) \cup l_1 \cup l_2 \right)  \right)$. Тогда $\int\limits_{\gamma} \frac{f(\xi) \dd{\xi}}{\xi - z} = 0 = \int\limits_{|\xi - z_0| = \eps} \frac{f(\xi)\dd{\xi}}{\xi - z} - \int\limits_{|\xi - z_0| = \delta} \frac{f(\xi) \dd{\xi}}{\xi - z}.$

    Тогда $\int\limits_{|\xi - z| = \delta} \frac{f(\xi) \dd{\xi}}{\xi - z} = 2\pi i f(z)$. Тогда оценим $\left| \int\limits_{|\xi - z_0| = \delta} \frac{f(\xi) \dd{x}}{\xi - z} \right| \le 2 \pi \delta \sup |f(\xi)|$.
\end{proof}
\begin{lemma}
   $f\!: \Omega \setminus \left\{ z_0 \right\} \to \CC$ --- голоморфная, $z_0$ --- полюс. Тогда  $\exists \eps > 0$,  $\vphi\!: B(z_0, \eps) \to \CC$ --- голоморфная,  $\vphi(z_0) \neq 0$,  $f(z) = (z - z_0)^{-d} \cdot \vphi(z), d \in \N$. 
\end{lemma}
\begin{proof}
   $h(z) = \frac{1}{f(z)}, h(z_0) = 0 \implies h(z) = (z-z_0)^{d} \cdot g(z), g(z_0) \neq 0$. $f(z) = \frac{1}{h(z)} = (z-z_0)^{-d} \frac{1}{g(z)} = \vphi(z)$.
\end{proof}
\begin{consequence}
    $f$ --- как в лемме, то  $\exists \eps > 0\!: \forall z \in B(z_0, \eps)$.

    $f(z) = \sum\limits_{n \ge -d} a_n (z-z_0)^n = a_{-d}(z-z_0)^{-d} + a_{-d + 1}(z-z_0)^{-d+1}+\ldots+a_{-1}(z-z_0)^{-1}+\psi(z)$, где $\psi(z)$ --- голоморфная.

    $f$ называется рядом Лорана. Все, что не  $\psi$ --- главная часть ряда Лорана.
\end{consequence}
\begin{proof}
    $f(z) = \frac{\vphi(z)}{(z-z_0)^d} = (z-z_0)^{-d} \sum\limits_{n \ge  0} b_n (z-z_0)^n$
\end{proof}
\begin{definition}
    $f\!: \Omega \setminus \left\{ z_0 \right\} \to \CC$ голоморфная, $z_0$ --- полюс, $f(z) = \sum\limits_{n \ge  -d}a_n(z-z_0)^n$. 

    Тогда \textbf{вычет $f$ в  $z_0$} --- $a_{-1}$, обозначение $\Res_{z_0} f = a_{-1}$.
\end{definition}
\begin{lemma}
    $Omega$ --- область,  $z_0 \in \Omega$,  $f\!: \Omega \setminus \left\{ z_0 \right\} \to \CC$ --- голоморфная, $z_0$ --- полюс, тогда, если $\eps > 0$ достаточно мало, то  $\int\limits_{|z - z_0| = \eps} f\left( z \right) \dd{z} = 2 \pi i \Res_{z_0} f$.
\end{lemma}
\begin{proof}
    $f(z) = \sum\limits_{n = -d}^{-1} a_n (z-z_0)^n + \psi(z), z \in B(z_0, \alpha \eps)$.

    Тогда $\int\limits_{\left| z-z_0 \right| = \eps} f(z) \dd{z} = \sum\limits_{n = -d}^{-1} \int\limits_{\left| z - z_0 \right| = \eps}a_n(z-z_0)^n \dd{z} + \int\limits_{\left| z-z_0 \right| = \eps}\psi(z) \dd{z} = 2 \pi i a_{-1}$
\end{proof}
\begin{lemma}
    $f\!: \Omega \setminus \left\{ z_0 \right\}$ --- голоморфная, $z_0$ --- полюс порядка $k$. Тогда:
     \[
         \Res_{z_0}f = \lim\limits_{z \to z_0} \frac{1}{(k-1)!}\left( \frac{\dd{}}{\dd{z}} \right)^{k-1}\left( (z-z_0)^k f(z) \right) 
    .\] 
\end{lemma}
\begin{proof}
    $f(z) = \sum\limits_{n \ge -k}a_n (z-z_0)^n \implies (z-z_0)^k f(z) = \sum\limits_{n \ge  0}a_{n - k}(z-z_0)^n \implies (z-z_0)^k f(z)$ голоморфна в $B(z_0, \eps)$ в том числе в $z_0$ и формула выше --- формула для коэффициентов ряда Тейлора. 
\end{proof}
\begin{definition}
    Пусть $\left\{ z_1, \ldots \right\} \subset \Omega$ --- дискретное подмножество $\Omega$. Тогда  $f\!: \Omega \setminus \left\{ z_i \right\} \to \CC$ называется мероморфной функцией в $\Omega$, если
     \begin{itemize}
        \item $f$ --- голоморфной,
        \item  $\forall k, z_k$ --- полюс  $f$.
    \end{itemize}
\end{definition}
\begin{lemma}
    $f, g\!: \Omega \to \CC$ --- голоморфная,  $g \not \equiv 0$, тогда  $\frac{f}{g}$ --- мероморфна в $\Omega$.
\end{lemma}
\begin{proof}
    Пусть $\left\{ z_1, z_2, \ldots \right\} \subset \Omega$ --- нули $g$, тогда  $\left\{ z_1, z_2, \ldots \right\}$ --- дискретное множество $\implies h = \frac{f}{g}$ задана и голомофорна в $\Omega \setminus \left\{ z_1, \ldots \right\} $.

    $z_k$ --- ноль порядка $d$ для  $g$, тогда
    \begin{itemize}
        \item если $f(z_k) \neq 0$, то локально $h(z) = \frac{1}{g(z) / f(z)} \implies \frac{1}{h(z)} = \frac{g\left(z  \right)}{f(z)}$ голоморфна в $z_0$ и равна нулю.
        \item $f(z_k)  0$, то пусть  $\widetilde{d}$ --- порядок нуля  $f$ в  $z_k$. Тогда локально  $g(z) = (z-z_k)^{d} \vphi(z), f(z) = (z-z_k)^{\widetilde{d}} \widetilde{\vphi}(z)$. $\frac{f(z)}{g(z)} = (z-z_k)^{\widetilde{d} - d} = \frac{\vphi(z)}{\widetilde{\vphi}(z)}$ --- голоморфная в $z_0$.
    \end{itemize}
\end{proof}
\begin{theorem}
    Любая мероморфная функция имеет вид $\frac{f}{g}$, $f, g$ --- голоморфная.
\end{theorem}
\begin{theorem}[Теорема о вычетах]
    Пусть $\Omega$ --- область,  $z_1, \ldots, z_n \in \Omega$. $f\!: \Cl \Omega \setminus \left\{ z_1, \ldots, z_n \right\} \to \CC$ --- непрерывная, голоморфная в $\Omega \setminus \left\{ z_1, \ldots, z_n \right\}$.

    Пусть $f$ имеет полюса в  $z_1, \ldots ,z_n$ или устранимые особенности. Тогда
    \[
        \int\limits_{\partial \Omega} f(z) \dd{z} = 2 \pi i \sum\limits_{k=1}^n \Res_{z_k}f
    .\] 
\end{theorem}
\begin{remark}
    $\frac{f(\xi)}{\xi - z}$ --- меромофрна в $B(z_0, \eps) \implies \int\limits_{\left| \xi - z_0 \right| } \frac{f(\xi) \dd{\xi}}{\xi - z} = 2\pi i \Res(\ldots) \implies$ формула Коши.
\end{remark}
\begin{proof}
    Картинка.
    $\Omega_\eps = \Omega \setminus \left( \bigcup_{k=1}^n \overline{B}(z_k, \eps) l_k \right), f$ --- голоморфная в $\Omega_\eps$.  $0 = \int\limits_{\partial \Omega_\eps} f(z) \dd{z} = \int\limits_{\partial \Omega} f(z) \dd{z} - \sum\limits_{k=1}^n \int\limits_{|z-z_k| = \eps} f(z) \dd{z} \implies \int\limits_{\partial \Omega} f(z) \dd{z} = \sum\limits_{k=1}^n \int\limits_{\left| z - z_k \right| = \eps} f(z) \dd{z} = 2 \pi i \sum\limits_{k=1}^n \Res_{z_k} f$.
\end{proof}
\Subsection{Принцип аргумента}
Пусть $z = r e^{i\theta}, r, \theta \in \R, r > 0$, тогда $\theta = \arg z$. $\arg z$ определен с точностью до  $2\pi$.
\begin{remark}
    $z\!:[a, b] \to \CC \setminus \left\{ \theta \right\}$ --- непрерывна, то $\exists r, \theta\!: [a, b] \to \R$ непрерывна и  $r(t) > 0 \forall t \in [a, b]$,  $z(t) = r(t) e^{i\theta(t)}$.
\end{remark}
\begin{example}
    Если $z$ параметризует окружность, то можно положить  $z(t) = e^{2 \pi}, z\!: [0, 2\pi] \to \CC, \theta(t) = t$.
\end{example}
$U \subset \CC \setminus \left\{ 0 \right\}$ --- односвязное, $z_0 = r_0 e^{i \theta_0} \in U$, тогда $\exists \log\!: U \to \CC$, такой что  $\log(z_0) = \log r_0 + i\theta_0, \frac{\dd{}}{\dd{z}} \log z = \frac{1}{z}$.

\begin{definition}
    $\Omega$ --- любая область,  $f\!: \Omega \to \CC$ голоморфная,  $f \not \equiv 0$, тогда логарифмическая  производная то $\left( \log f\left(z  \right)  \right)' \coloneqq \frac{f'(z)}{f(z)}$.
\end{definition}
\begin{statement}
    $(\log f)'$ --- это мероморфная функция в $\Omega$, все полюсы  $\left( \log f \right) '$ простые и соответствуют нулям $f$. Если  $f(z_0) = 0$, то $\Res_{z_0}(\log f)' = \Ord_{z_0} f$ --- порядок нуля. 
\end{statement}
\begin{proof}
    Пусть $f(z_0) \neq 0 \implies \frac{f'(z)}{f(z)}$ --- голоморфная в окрестности $z_0 \implies (\log f)'$ голоморфна в $\Omega \setminus \left\{ z \!: f(z) = 0 \right\}$.

    Пусть $f(z_0) = 0$, напишем  $f\left(z  \right) = (z-z_0)^{d} g(z)$, где $d = \Ord_{z_0} f, g(z_0) \neq 0$.
\end{proof}
\begin{theorem}[Принцип аргумента]
    $\Omega$ --- односвязное, ограниченное,  $\partial \Omega$ --- кусочно гладкая,  $f$ --- голоморфная в окрестности  $\Cl \Omega$ (то есть $\exists \Omega' \supset \Cl \Omega$и  $f\!: \Omega' \to \CC$ --- голомофорная) и $f(z) \neq 0\, \forall z \in \partial \Omega$. Тогда  \[
        \int\limits_{\partial \Omega} \left( \log f(z) \right)' \dd{z}] = 2\pi i \sum\limits_{z \in \Omega\!: f(z) = 0} \Ord_{z}f \eqqcolon 2 \pi i \text{\# нулей в }f\text{ с учетом кратности}
    .\] 
\end{theorem}
\begin{proof}
    $\int\limits_{\partial \Omega} \left( \log f(z) \right)' \dd{z} = 2 \pi i \sum\limits_{z\text{--- полюс}} \Res_{z} (\log f)' = 2\pi i \sum\limits_{z\!: f(z) = 0} \Ord_z f$
\end{proof}

Пусть $z\!: [a, b] \to \CC$ --- параметризация $\partial \Omega$, пусть также  $f(\Omega') = \Cl$. Тогда  $\Log f(z)$ корректно определена, $(\log f(z))' = \frac{\dd{}}{\dd{z}} \Log f(z)$.

$\int\limits_{\partial \Omega} (\log f(z))' \dd{z} = \int\limits_{\partial \Omega} \frac{\dd{}}{\dd{z}} \Log f(z) \dd{z} = \int\limits_{a}^{b} \frac{\dd{}}{\dd{z}}\Log f(z(t)) z'(t) \dd{t} = \int\limits_{c}^{b} \frac{\dd{}}{\dd{t}} \left( \Log f(z(t)) \right) \dd{t} = \int\limits_{a}^{b} \frac{\dd{}}{\dd{t}} \left( \log \left| f(z(t)) \right| + i \arg f(z(t)) \right) \dd{t} = \log \left| f(z(t)) \right| - \log |f(z(a))| + i \arg f(z(b)) - i \arg f(z(a))$

\begin{example}
    $\Omega = \mathbb{D}, f(z) = z^n, z\!: [0, 2\pi] \to \CC, z(t) = e^{it}$. Тогда $f(z(t)) = e^{i n t}, \theta(t) = nt$ 

    $\int\limits_{\partial \Omega} \left( \log f(z) \right)' \dd{z} = i(\theta(2\pi) - \theta(0)) = n 2 \pi i$.
\end{example}
\begin{theorem}[Теорема Руше]
    Пусть $\Omega$ --- односвязная область, ограниченная  $\partial \Omega$ --- кусочно гладкий путь.$f, g$ --- голоморфная в окрестности  $\Cl \Omega$, $\forall z \in \partial \Omega\quad |f(z)| > |g(z)|$.

    Тогда  \# нулей $f$ в  $\Omega$ с учетом кратности равно количеству нулей  $f+g$ в  $\Omega$ с учетом кратности.
\end{theorem}
\begin{proof}
    $t \in [0, 1]$. Рассмотрим  $\Phi(t) = \frac{1}{2 \pi i} \int\limits_{\partial \Omega} (\log (f+tg) (z))' \dd{z} = \frac{1}{2 \pi i}\int\limits_{\partial \Omega} \frac{f'(z) + tg'(z)}{f(z) + tg(z)} \dd{z}$.

    \begin{enumerate}
        \item $\Psi(z, t)\!: \partial  \Omega \times [0, 1] \to \CC$.  $\Psi(z, t) = \frac{f'(z) + tg'(z)}{f(z) + tg(z)}$ непрерывна $\implies \Phi(t) = \frac{1}{2 \pi i} \int\limits_{\partial \Omega} \Psi(z, t) \dd{z}$ непрерывна.
        \item $\forall t \in [0, 1], \Phi(t) \in \Z$ по теореме выше.
    \end{enumerate}

    Из 1 и 2 следует, что $\Phi(t) \equiv n, n \in 'Z$. Но  $\Phi(0) =$ количество нулей  $f$ в  $\Omega$ с учетом кратности, а  $\Phi(1)$ --- количество нулей  $f+g$ в  $\Omega$ с учетом кратности.
\end{proof}
\begin{theorem}
    $\Omega$ --- область,  $f\!: \Omega \to \CC$ --- голоморфная непостоянная, тогда $\forall z_0 \in \Omega, \delta > 0\!: \overline{B}(z_0, r) \subset \Omega \exists \delta > 0\!: f(B(z_0, r)) \supset B(f(z_0), \delta)$.
\end{theorem}
\begin{proof}
    Немного уменьшив $r$ мы можем добиться того, чтобы  $|f(z) - f(z_0)| \neq 0\quad \forall z \!: |z-z_0| = r > 0$.

    Возьмем $\delta = \min\limits_{z \in C_r(z_0)} |f(z) - f(z_0)| > 0$. Пусть $\lambda \in B(f(z_0), \delta)$, тогда по теореме Руше. $1 \le $ количество нулей $f(z) - f(z_0)$ в $B(z_0, r)$ с учетом кратности и это равно числу нулей $f(z) - f(z_0) - \lambda$ в том же шаре.

    Тогда $\exists z \in B(z_0, r)\!: f(z) = f(z_0) + \lambda$, такая что $\lambda \in B(0, \delta)$ произв., имеем  $f(B(z_0, r)) \supset B(z_0, \delta)$.
\end{proof}
\begin{consequence}
   Пусть  $\Omega$ --- ограниченная область,  $f\!: \Cl \Omega \to \CC$ непрерывна,  $f$ голоморфная в  $\Omega$. Тогда
   \begin{enumerate}
       \item $\sup\limits_{z \in \Omega} |f(z)| = \max\limits_{z \in \Cl \Omega} |f(z)| = \max_{z \in \partial \Omega} |f(z)|$
       \item Если $\exists z_0 \in \Omega\!: |f(z_0)| = \sup\limits_{z \in \Omega} |f(z)| \implies f \equiv const$.
   \end{enumerate}
\end{consequence}
\begin{proof}
    Пусть $f$ не постоянна,  $z_0 \in \Omega, r > 0\!: \overline{B}(z_0, r) \subset \Omega$. Тогда $\exists \delta > 0\!: f(B(z_0, r)) \supset B(f(z_0), \delta) \implies \exists z \in B(z_0, r)\!: |f(z)| > |f(z_0)| \implies |f(z_0)| < \sup\limits_{z \in \Omega} |f(z)| \implies 2$.

    Чтобы увидеть $1$, заметим, что  $\exists z_0 \leftarrow \Cl \Omega\!: |f(z_0)| = \max\limits_{z \in \Cl \Omega} |f(z)|$. Если $z_0 \in \Omega$, то ?! с рассуждениями выше.
\end{proof}
